
\documentclass[12pt,a4paper]{article}
\usepackage{draftwatermark}

% \linenumbers
\SetWatermarkText{E L DALISTAN}
\SetWatermarkScale{3}

%%%%%%%%%%%%%%%%%%%%%%%%%%%%%%%%%%%%%%%%%%%%%%%%%%
\setlength{\topmargin}{0mm}
\setlength{\headheight}{0mm}
\setlength{\headsep}{0mm}
\setlength{\oddsidemargin}{0mm}
\setlength{\textwidth}{165mm}
\setlength{\textheight}{247mm}
\setlength{\parindent}{6pt}
\setlength{\parskip}{0pt}
%%%%%%%%%%%%%%%%%%%%%%%%%%%%%%%%%%%%%%%%%%%%%%%%%%

\begin{document}
\noindent \textbf{CE297 - MX1} \\
\textbf{Name:} \noindent Enrico Miguel Dalistan \\
\textbf{Student No.} \noindent 2011-79757\\

\bigskip
\bigskip
\bigskip

\noindent \textbf{Brief Report}\\

\bigskip

\noindent This Machine Exercise features the Conjugate Gradient (CG) method,
an algorithm which uses iteration for solving systems of linear equation.
The implementation is written in Python 3.9 with no third-party library used
(such as NumPy). However, common helper functions are written separately
(matrix and vector operations, under \textit{/helpers} folder;
subject to further testing for bugs).\\

\noindent In addition, input vectors and matrices are processed also by the helper
functions (\textit{i.e., vectors.py and matrices.py, respectively}) from separate files*
(\textit{.dat, .out, .txt, .csv**}) which directories are harcoded into the script.
Hence, the actual implementation of the CG method was reduced to the actual algorithm
while the vector and matrix operations are abstracted away.\\

\bigskip

\noindent \textbf{Usage}\\

\noindent \textit{python3 conj\_grad.py}\\
\noindent returns the solution vector \textit{(x)}\\

\noindent Input matrix and vector are read from a separate file, only the directories
are specified.\\

\bigskip

\noindent Notes:\\
\noindent * files are read by lines hence delimiters are important in properly reading data.
Matrices can be delimited by spaces, tabs or commas while vectors should be written properly
in a column without trailing whitespaces\\
\noindent ** .csv files are only for matrices

\bigskip

\noindent \textbf{References}\\

\noindent [1] Shewchuk, J.R. (1994). An Introduction to the Conjugate Gradient Method Without the Agonizing Pain.
\textit{Technical Report}. Carnegie Mellon University, USA.

\bigskip
\bigskip
\bigskip

\begin{center}
    \textbf{** END **}
\end{center}
\end{document}