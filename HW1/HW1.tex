\documentclass[12pt,a4paper]{article}
\usepackage{hyperref}
%{setspace}
\usepackage{animate}
\usepackage[dvips]{graphicx}
\usepackage{graphics}
\usepackage{xcolor}
\usepackage{amsmath}
\usepackage{lineno}
\usepackage{draftwatermark}


\linenumbers
\SetWatermarkText{P E QUINAY}
\SetWatermarkScale{3}


%%%%%%%%%%%%%%%%%%%%%%%%%%%%%%%%%%%%%%%%%%%%%%%%%%
\setlength{\topmargin}{0mm}
\setlength{\headheight}{0mm}
\setlength{\headsep}{0mm}
\setlength{\oddsidemargin}{0mm}
\setlength{\textwidth}{165mm}
\setlength{\textheight}{247mm}
\setlength{\parindent}{6pt}
\setlength{\parskip}{0pt}
%%%%%%%%%%%%%%%%%%%%%%%%%%%%%%%%%%%%%%%%%%%%%%%%%%


\begin{document}

\begin{center}
  \textbf{University of the Philippines Diliman\\ College of Engineering -- Institute of Civil Engineering} \\

  \textbf{CE 297 Finite Element Method for Structural Dynamics \\ 1$^{st}$ Semester A.Y. 2021-2022} \\

  \bigskip
  \bigskip

  \textbf{Homework 1:}

  \textbf{TeX Document Editing Practice}
\end{center}

\bigskip

\noindent Start Date: September 17, 2021 \\
\noindent Due Date: September 24, 2021 (send through email: pbquinay2[at]up.edu.ph)\\
\noindent To submit: CE297\_HW1\_\{Your Surname\}.tex file and the output PDF file\\

\noindent \textbf{Instructions:}

Use any TeX Editor (TeXworks, TeXstudio, MiKTeX, TeXmaker, Overleaf) to work on the following:

\begin{enumerate}
  \item Write your background related to FEM (related courses or trainings taken previously, etc.), if any.

  \item Write your interests related to FEM and/or dynamic problem (target practical applications, thesis, or dissertation topic, etc.)?

  \item Create a three-column table listing your favorite (finished) M.S. and/or PhD courses with corresponding Course Code and Title, and Year Taken (use the standard TeX command on how to create a table); for example, see Table \ref{sampletablelabel}

  \item Create a figure of any image or photo related to dynamic problem, with caption (use the standard TeX command on how to create a figure; also cite the source of image or photo, if applicable; for example see Figure \ref{samplelabel})

  \item (Optional) Create an animation, with caption (use the standard TeX command on how to create a figure; also cite the source of image or photo, if applicable; for example see Figure \ref{sampleanim}).

  \item Write your favorite equation in TeX format (Example, See Eq. \ref{labeleq1}).

  \item Enumerate your free days/available schedules for make-up class(es) (for example, Wednesdays, 6-9pm) (use the standard TeX command on how to enumerate a list).



\end{enumerate}

\noindent Note 1: Most queries in using TeX can be found in the Web, so there is no need to buy a book. \\

\noindent Note 2: You can also play with formatting, such as using \textit{italics}, \textbf{bold}, and \textcolor{red}{colored fonts}. \\

\noindent \textbf{Settings:} \\
Paper size: A4 \\
Font size: 12 pt \\
Output format: PDF\\

\bigskip

\hrulefill

\newpage

\noindent Available schedules for consultation:
\begin{enumerate}
  \item Tue, Thu: 1:00PM-5:00PM
  \item Wed, Fri: 9:00AM - 12:00NN
\end{enumerate}





\begin{table} [htb]
  \begin{center}
    \caption{My Courses}
    {\begin{tabular}[t]{ccc}
        \hline
        \textbf{Course} & \textbf{Course Title}                   & \textbf{Year}  \\
        \textbf{Code}   &                                         & \textbf{Taken} \\    \hline
        ES204           & Numerical Methods in Engineering        & 2019           \\
        CE226           & Structural Dynamics                     & 2020           \\
        CE257           & Discrete Methods of Structural Analysis & 2021           \\     \hline
      \end{tabular}}
    \label{sampletablelabel}
  \end{center}
\end{table}




\begin{equation}
  (\mathbf{K}+\frac{2}{\Delta t}\mathbf{C}+\frac{4}{\Delta t^2}\mathbf{M})\mathbf{u}^{n+1}=(\frac{2}{\Delta t}\mathbf{C}+\frac{4}{\Delta t^2}\mathbf{M})\mathbf{u}^n+(\mathbf{C}+\frac{4}{\Delta t}\mathbf{M})\mathbf{v}^{n}+\mathbf{Ma}^{n}+\mathbf{f}^{n+1},
  \label{labeleq1}
\end{equation}



\begin{figure}[!h]
  \begin{center}
    \scalebox{1.25}{\includegraphics{./figure/bridge}}
    \caption{Sample FEM meshed model}
    \label{samplelabel}
  \end{center}
\end{figure}



\begin{figure}[!h]
  \begin{center}
    \animategraphics[controls,loop,width=6in]{3}{./anim/disp}{1}{11}
    \caption{Sample animation}
    \label{sampleanim}
  \end{center}
\end{figure}


\bigskip

\bigskip

\begin{center}
  \textbf{- End of Instructions for Homework 1 -}
\end{center}

\end{document}
